%!TEX root = ../username.tex
\chapter{Real Acoustic Spaces}
\hspace*{-0.15cm}This chapter will cover impulse responses and their use in artificial reverberation. It will begin with the mathematical background of rooms and how their echo is measured. Next, several methods of measuring a room's impulse response will be discussed. Finally, the measurements taken from an acoustic space will be provided and the results analyzed.

\section{Mathematical Background}
With the tools [], it is easy to presume that the []. It comes naturally the one will need the impulse response of a room in order to have something to convolve with. While several can be found online, measuring the impulse respones oneself []. Unlike purely artificial means where the delay lines of the same design can be tuned to one's taste, convolving with an impulse response should result in the same output

\section{Measuring the Impulse Response}
While any space could have been used to measure an impulse response, there were practical limitations that prevented just any room from being measured. A potential room needed to be both acoustically interesting, but also have the eqipment necessary to perform accurate measurements. For these reasons, Gault Recital Hall - located in Scheide Music Center - was the room chosen. To measure the impulse response of Gault Recital Hall, a logarithmic sine sweep of the room was performed, spanning the range of typical human hearing. These were played from Bag End TA6002-I speakers located approximately halfway up the room on stage left and stage right, with an array of Infra Subwoofers in the middle of the ceiling. There, the response of the sweep was recorded with a Zoom XYH-6 X/Y Capsule placed roughly in the center of the room. Several measurements were taken for redundancy, in case any one recording did not suffice or included unwanted noise.

From these recordings, the raw audio files were then processed with ReaVerb - an application included with Reaper that allows for impulse response generation by providing the raw audio and the inverse logarithmic sweep.

\section{Results of a Sinusoid Sweep}
