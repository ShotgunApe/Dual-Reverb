%!TEX root = ../username.tex
\chapter{Conclusion and Future Work}
\hspace*{-0.15cm}This barely scratches the surface. While the topics mentioned in this thesis cover the fundamentals of artificial reverberation, there are many different designs and improvements that can be made over the program as it is described here. Schroeder reverberators are considered a specific case to the more generalized structure known as a \textit{Feedback Delay Network (FDN)} \cite{schlecht2016lossless}. A modernized artificial reverberator will use this type of structure with a diffuser step, should one take the route of a purely artificial design. Through this design, a reverberator would split the audio into several channels - similar to that of Comb Filters in parallel - but would manipulate the signal by distributing the amplitude amongst the matrix of channels to provide a more diffuse sound \cite{writeReverb}. This is just one area that is possible to improve upon.

Several improvements can be made during the prediction and measurement of Impulse Responses, as well. The Absorption Coefficients found in Table 5.1 were chosen as what best approximated the materials found in the room - a better selection of coefficients would likely yield a more accurate result. Along with this, the simplification of the room geomety would introduce some amount of error in the prediction as well - a better prediction would take each surface into account, including steps and curves of the performance walls. When recording, better practice would be to include a marker to better align the raw audio files with the impulse to deconvolve with - this is likely the culprit for the noise found in Figures 5.2 and 5.4.

The code itself can be improved upon, as well. In its current state, the Convolution approach includes several glitches in the audio output, which is likely due to the number of calculations required for the length of the Impulse Response. While building a ``Release'' build of the application \textit{significantly} improves the latency of this approach, it still can be improved upon with better tuning and usage of JUCE's DSP class. Similarly, the code behind the Schroeder Comb Filter approach can be improved through less copying of buffers - there is likely a more efficient solution that involves less usage of JUCE's classes that would allow for a larger number of filters in parallel before performance takes a hit, similar to what Freeverb was able to accomplish.

Despite this, this thesis provides a groundwork for any future audio engineer wishing to develop this type of application. I encourage anyone wishing to learn more to search and read through the citations used, as they contain a wealth of knowledge that this thesis only uses a fraction of. Anyone wishing to view the code can find it in Appendix A and online \cite{CODE}. While the topics and filters used are specifically in the context of artificial reverberation, these same techniques can be applied to a wide variety of other effects, including distortion, vocoding, and autotune.
