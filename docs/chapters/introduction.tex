%!TEX root = ../username.tex
\chapter{Introduction}\label{intro}
%holy jank but i figured it out
\hspace*{-0.155cm}Really meaningful songs take the listener on a journey. Regardless of genre, music has the ability to connect people together of a wide variety of backgrounds and experiences. In one way, a composer chooses different notes in a piece to follow particular harmonic and melodic conventions. Meeting these conventions and subverting them in interesting ways is part of what makes music, music. In another way, much like a song's chord progression, the timbre of a sound itself can create greater depth and immersion in one's production to help bring the listener from point ``A'' to point ``B''. Several tools are at a producer's disposal to modify the color and shape of a particular sound in real time. For example, distortion can be added to a guitar, or delay can be added to a singer's vocals. Reverb is another such essential tool at the producer's disposal used in music production today.

The study of artificial reverberation goes back over a hundred years. During this time, the methods used to artificially reverberate sound has shifted from ad hoc methods to carefully crafted tools and programs. Today, the majority of programs are created for use under specific host applications - these are known as \textit{Digital Audio Workstations}. By easily handling input and output of other programs which manipulate a given signal, one can perform complex sound design by utilizing a combination of several programs in series.

What follows is a study into the development and execution of artificial reverberation in C++. To better understand how reverberation is used in the context of music production, we will begin with a historical view of this tool. Next, we will look at the programs and software that producers use to create music today, and in what ways they use reverberation. Specifically, this will include discussion of Digital Audio Workstations and their associated programs. After, we will consider the digital signal processing theory required to create these tools. This will include how to represent sound in a digital manner and in what ways the computer processes this sound. Finally, we will discuss various methods of persisting sound with a short impulse. We will look at two different methods in particular: \textit{Feedback Delay Networks} and \textit{Convolution Algorithms}. In looking at these two common methods, someone inexperienced in this topic should not only gain an understanding in how artificial reverberators are made, but also be able to implement one themselves.
