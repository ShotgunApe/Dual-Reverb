%!TEX root = ../username.tex
\chapter{Introduction}\label{intro}
%holy jank but i figured it out
\hspace*{-0.155cm}Really meaningful songs take the listener on a journey. Regardless of genre, the ability that music has to create a [] that provides a way to connect with people. If music is a storytelling medium, reverb is a storytelling device; much like a song's chord progression, the timbre of a sound can create greater depth and immersion in one's production to help bring the listener from point ``A'' to point ``B''. Reverb is an essential tool used in music production today.

What follows is a study into the development and execution of artificial reverberation in C++. In looking at the two most common methods, someone inexperienced in this topic should not only gain an understanding in how artificial reverberators are made, but also be able to implement one themselves. To better understand how reverberation is used in the context of music production, we will begin with a historical view of this tool. Next, we will look at the programs and software that some producers use to create music today, and in what ways they use reverberation. Specifically, this will include discussion of Digital Audio Workstations and their []. After, we will consider the digital signal processing theory required to create these tools. [] Finally,
