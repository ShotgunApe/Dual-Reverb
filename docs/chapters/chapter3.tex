%!TEX root = ../username.tex
\chapter{Artificial Reverberation Design}
\hspace*{-0.15cm}This chapter will begin with a brief overview of digital signal processing to understand artificial reverberation algorithms. After, various methods of reverberation will be explored; specifically going into detail how they work and their corresponding signal flow graphs.

\section{Filters \& Signal Flow Graphs}
Thus far, signals have been represented as a function with respect to time. This is not the only manner in which one can visualize audio, however - by computing the \textit{Fourier Transform} of a function, it can be graphed as a function with respect to \textit{frequency}, as opposed to time. At a high level, the Fourier Transform deconstructs a signal to its sinusoids; this was discussed in Chapter 3. However, to compute the Fourier Transform, the previously discussed mathematical representation of sound will need to be reevaluated.

Phase shifted sinusoids are ambiguous as to how they can be represented mathematically. That is, the same sinusoid can be represented as a function of sine \textit{or} cosine. To perform calculations useful to the programmer, one needs a representation of a sinusoidal wave that removes this ambiguity. The \textit{complex sinusoid} does exactly that by encapsulating both functions under the complex plane \cite{pirkle2019designing}:

\begin{defn}[Definition of a complex sinusoid]\label{def-complex}
	\begin{equation}\label{eq-complex)}
	e^{j\omega t} = cos(\omega t) + j sin(\omega t)
\end{equation}\end{defn}

where $\omega$ is radians, $t$ is time, and $j$ is $ \sqrt{-1}$. With this definition, the Fourier Transform $X(\theta)$ may be calculated from a signal $x(t)$.

\begin{defn}[Definition of the Fourier Transform \cite{MDFTWEB07}]\label{def-Cont-FT}
	\begin{equation}\label{eq-Cont-FT)}
	X(\omega) = \int_{-\infty}^\infty x(t)e^{-j\omega t}dt, \quad \omega \in (-\infty, \infty)
\end{equation}\end{defn}

In a digital setting, this process is computed via the summation of samples from a given input buffer:

\begin{defn}[Definition of the Discrete Fourier Transform \cite{MDFTWEB07}]\label{def-disCont-FT}
	\begin{equation}\label{eq-disCont-FT)}
	X(\omega_k) = \sum_{n=0}^{N-1} x(t_n)e^{-j\omega_k t_n}, \quad k = 0,1,2,..., N - 1
\end{equation}\end{defn}

Revisiting the Nyquist-Shannon Sampling Theorem, one may graph the bandlimited signal $F_{max}$ and Nyquist Rate $F_s$ with respect to frequency using a DFT. Such an example can be seen below:

\begin{figure}[h] % [h] used to prevent {figure} from doing weird positioning
	\begin{center}
		\fbox{
		\begin{tikzpicture}
            \begin{axis} [
			axis x line = middle, % The x axis should go through the origin
			axis y line = middle,
			xmin = -7,
			xmax = 7,
			ymin = -1.5,
			ymax = 7,
			xlabel = \(f\),
			ylabel = {\(X(f)\)},
			height = 7cm,
			width = 13cm,
			xtick distance=6,
			ytick distance=-10,
			xticklabel = \empty
			]
			\addplot[very thick, red][domain=-3:-2]{+0.8*x^3+7.2*x^2+22.5*x^1+24.3*x^0};
			\addplot[very thick, red][domain=-2:-1]{+-2.4*x^3+-12*x^2+-15.9*x^1+-1.3*x^0};
			\addplot[very thick, red][domain=-1:0]{+2.9*x^3+3.9*x^2+8.608e-62*x^1+4*x^0};
			\addplot[very thick, red][domain=0:1]{+-2.9*x^3+3.9*x^2+8.608e-62*x^1+4*x^0};
			\addplot[very thick, red][domain=1:2]{+2.4*x^3+-12*x^2+15.9*x^1+-1.3*x^0};
			\addplot[very thick, red][domain=2:3]{+-0.8*x^3+7.2*x^2+-22.5*x^1+24.3*x^0};
			\node[draw] at (1000, 70) {$F_{max}$};
			\node[draw] at (400, 70) {$F_{max}$};
			\node[draw] at (1300, 70) {$F_{s}$};
			\node[draw] at (100, 70) {$F_{s}$};
			\end{axis}
		\end{tikzpicture}
		}
		\caption{An example of a Fourier Transform taken from a function.}
	\end{center}
\end{figure}

Recall the way a digital audio workstation functions. Generated sound is sent through various mixer channels to be modified before combining in the master channel. While various effects are available to manipulate sound in the mixer, exactly \textit{how} they work is behind closed doors. Musicians may be familiar with tools such as ``lowpass filters'' and ``highpass filters'' - audio engineers expand what exactly a ``filter'' is to include \textit{all} types of different effects. With this in mind, a \textit{filter} can be defined as a medium through which a signal enters as an input and exits as an output \cite{FILTERS07}. It is a type of \textit{black box system} whose inner workings modify the sound in some way; it is inclusive of any type of effect one can think of. By extension, a \textit{digital filter} is a type of filter that operates on digital signals in the form of a computation. This computation can take the form of an equation, or a snippet of code in a computer.

Consider the following graph:

[show graph of low-pass filter and describe the differences in representation of sound in time vs frequency]

It is worth emphasizing that this is a different representation of the previous sound, as they are a graph of frequency to amplitude as opposed to the waveform over time (reword this). In fact, [] For this reason [].

[maybe talk about solving for a FFT of a specific waveform to describe that section? maybe introduce complex numbers here?]

\textit{Signal Flow Graphs} are used to visualize how filters manipulate a provided input and output \cite{FILTERS07}. They represent a system of linear equations that compute an output signal based on past and present input signals and past output signals. For example, the following graph describes one type of filter known as a \textit{comb filter}:

[insert comb filter here]


This comb filter can be represented through code, as well. To implement this, a data structure known as a \textit{circular array} will need to be implemented.

\section{The Schroeder Comb Filter Approach}

%\section{Feedback Delay Networks}
%Feedback Delay Networks are described as a generalized comb filter. As opposed to having a single delay line in the system, a number %of delay lines are used in parallel each containing some combination of the output signal \cite{PUCKE}. A typical FDN is built with %several delay lines \textit{N}, each having a length provided by the equation [].

%A typical FDN is represented by the following relation:

%\begin{defn}[A typical FDN in the time domain \cite{OnLossless}]\label{def2}
%	\begin{equation}\label{nextdef(t)}
%	y(n)=\sum_{i=1}^{N} c_i s_i (n) + d x (n)
%	\end{equation}
%	\begin{equation}
%	s_i(n + m_i)=\sum_{j=1}^{N} a_{ij} s_j (n) + b_i x (n)
%\end{equation}\end{defn}

%where $x(n)$ and $y(n)$ are the input and output values and []. Figure [] provides the Signal Flow Graph for Definition 4.1.

\section{Convolution}
