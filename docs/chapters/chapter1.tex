%!TEX root = ../username.tex
\chapter{Background}
\hspace*{-0.155cm}This chapter will provide the backgorund information necessary to understanding what artificial reverberators are, why their use is important, and the different ways they are used by musicians today.

\section{Reverberation Information}
The use of artificial reverberators is that of a tool to producers and musicians. They provide additional color to an otherwise dry sound by emulating how sound might travel throughout a large room. Prior to the early 1960s, ``artificial reverberators'' were physical devices - that is, something that took input from an electromechanical transducer and manipulated the sound using a variety of physical methods. Some such devices used metal plates, springs, and oil canisters - each providing its own unique sound and characteristic that some musicians liked (and others did not). In 1962, M. R. Schroeder introduced a number of methods to digitally recreate this effect without its disadvantages. This by no means replaced the physical devices, however - Schroeder's paper simply marked some of the first algorithms used to generate artificial reverberation by purely electronic means. These same algorithms are used in several programs today.

In addition to these reverberation methods, this section will cover how artificial reverberation is used by musicians today. Common music technology methods will be discussed so that one can understand in what context these algorithms are used. This will include digital audio workstations and virtual studio technology.

\section{Reverberation Methods}

\subsection{Echo Chambers}

\subsection{Electromechanical Devices}
\subsubsection{Spring Reverberators}
\subsubsection{Plate Reverberators}
Completely independent of spring reverberators, plate reverberators were concieved of by utilizing a similar principle.

\subsubsection{Oil-Can Reverberators}
\subsection{Analog Reverberation Circuits(?)}
\section{Digital Audio Workstations}
\subsection{Usage of DAWs}
\subsection{Virtual Studio Technology}
