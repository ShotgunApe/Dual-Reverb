%!TEX root = ../username.tex
\chapter{Background}
\hspace*{-0.155cm}This chapter will provide the backgorund information necessary to understanding what artificial reverberators are, why their use is important, and the different ways they are used by musicians today.

\section{Reverberation Information}
The use of artificial reverberators is that of a tool to producers and musicians. They provide additional color to an otherwise dry sound by emulating how sound might travel throughout a large room. Prior to the early 1960s, ``artificial reverberators'' were physical devices - that is, something that took input from an electromechanical transducer and manipulated the sound using a variety of physical methods. Some such devices used metal plates, springs, and oil canisters - each providing their own unique sound and characteristics. In 1962, M. R. Schroeder introduced a number of methods to digitally recreate this effect without additional color. This by no means replaced the physical devices, however - Schroeder's paper simply marked some of the first algorithms used to generate artificial reverberation by purely electronic means. These same algorithms are used in several programs today.

\section{Reverberation Methods}

\subsection{Echo Chambers}

\subsection{Electromechanical Devices}
\subsubsection{Spring Reverberators}
\subsubsection{Plate Reverberators}
\subsubsection{Oil-Can Reverberators}
\subsection{Analog Reverberation Circuits}
\section{Digital Audio Workstations}
\subsection{Usage of DAWs}
\subsection{Virtual Studio Technology}
