%!TEX root = ../username.tex
\chapter{Background}
\hspace*{-0.155cm}This chapter will provide the backgorund information necessary to understanding what artificial reverberators are, why their use is important, and the different ways they are used by musicians today.

\section{What is Reverb?}
Reverb can be broadly defined as the persistence of sound after its initial impulse response. This can take the form of sound waves bouncing off the walls of a room, for example. It provides a texture that is considered ``wet'' and washes out the sound of an otherwise clear tone. In small amounts, it can be tasteful and enhance the timbre of an instrument. In large amounts, it can overpower the initial tone and can disrupt other elements of the music, such as rhythm.

Consider an instrument with a short impulse response:

\begin{figure}[h] % [h] used to prevent {figure} from doing weird positioning
	\begin{center}
		\fbox{
		\begin{tikzpicture}
			\begin{axis} [
				axis x line = middle, % The x axis should go through the origin
				no markers,
				xlabel = \(t\),
				ylabel = {\(f(t)\)},
				height = 5cm,
				width = 12cm,
				xmin=0,
				xtick distance=0.2,
				]
				\addplot [
					red,
					thick
				] table [x=step,y=wav,col sep=comma] {test.csv};
			\end{axis}
		\end{tikzpicture}
		}
		\caption{A waveform of a snare drum.}
	\end{center}
\end{figure}

By adding reverb, this same instrument can sound as if it were performed in a large room:

\begin{figure}[h] % [h] used to prevent {figure} from doing weird positioning
	\begin{center}
		\fbox{
		\begin{tikzpicture}
			\begin{axis} [
				axis x line = middle, % The x axis should go through the origin
				no markers,
				xlabel = \(t\),
				ylabel = {\(f(t)\)},
				height = 5cm,
				width = 12cm,
				xmin=0,
				xtick distance=1,
				]
				\addplot [
					red,
					thick
				] table [x=step,y=wav,col sep=comma] {test2.csv};
			\end{axis}
		\end{tikzpicture}
		}
		\caption{A waveform of a snare drum with reverberation.}
	\end{center}
\end{figure} %fix to compress waveform to -1 and 1

Prior to the early 1960s, ``artificial reverberators'' were physical devices - that is, something that took input from an electromechanical transducer and manipulated the sound using a variety of physical methods. Some such devices used metal plates, springs, and oil canisters - each providing their own unique sound and characteristics \cite{FiftyYears}. In 1961, M. R. Schroeder introduced a number of methods to digitally recreate this effect without additional color \cite{schroeder1961natural}. This by no means replaced the physical devices, however - Schroeder's paper simply marked some of the first algorithms used to generate artificial reverberation by purely electronic means. These same algorithms are used in several programs today.

\section{How is Reverb Used?}
Reverb is most commonly used in \textit{Digital Audio Workstations}, applications which are used to produce music. Nearly every song one hears on the radio utilizes some kind of music application which allows one to record, edit, and manipulate several audio sources and recordings.
%\subsection{Usage of DAWs}
%\subsection{Virtual Studio Technology}


\section{Reverberation Methods}
Spring reverberators work by [].
%\subsection{Echo Chambers}
%\subsection{Electromechanical Devices}
%\subsubsection{Spring Reverberators}
%\subsubsection{Plate Reverberators}
%\subsubsection{Oil-Can Reverberators}
%\subsection{Analog Reverberation Circuits}
